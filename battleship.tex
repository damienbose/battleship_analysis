\documentclass[11pt]{article}
\usepackage[left=3cm,right=3cm,top=3cm,bottom=3cm]{geometry}
\usepackage{mathtools}
\usepackage{multicol}

%% (other macros, packages and definitions go here)

\newtheorem{definition}{Definition}

\begin{document}

\title{Analysis of the Battleship Game}
\author{Damien R. Bose}
\date{21/02/2023}

\maketitle

\section{Introduction}

This article will try and analyse the game of battleship. Our goal is to come up with a strategy for optimal (or close to optimal) play, inspired by an article by Anthony Rochford's analysis\cite{rochford}. However, I try to formalise these ideas in my own way. Furthermore, I try pay more attention to about the behaviour of the opponent. 

\section{Rules of the Game}

\section{Modelling the Game}

We first come to the question of modelling the opponents board. First, however, note


Notice that the game lies on a $10 \times 10$ grid. Hence, there are 100 tiles.

Therefore, we can represent the state of the opponents entire board at time $t$ as a vector of random variables corresponding to each tile.

\begin{definition}
 $B_t =(P(X_t^0 = 1),...,P(X_t^{99} = 1))$, where $P(X_t^i = 1)$ is the probability that $i\textsuperscript{th}$ the ith tile has a ship at time $t$.
\end{definition}

Notice that $X^i \sim Ber(\theta^i)$. Hence $P(X^i = 1) = \theta^i$ and determining the $\theta^i$ will allow us to choose the tiles most likely be a ship.

Furthermore, $B_0$ is initial placement of the battle ships on the opponents board. The distribution of $B_0$ is determined by the opponent. 

Now lets not forget out goal: be the first to hit all 17 boat tiles of the opponent. 

\begin{definition}
 $T_s$ is the number of moves/time it takes for us to do so with strategy s.
\end{definition}

We can therefore quantify the performance of our strategy using $E[T_s]$. This is not be the entire picture because winning is dependent on the performance of the other player as well. First, the initially

For example lets say player 1 has strategy $s_1$ and player 2 has strategy $s_2$, and $E[T_{s_1}] = E[T_{s_2}|B_0]$. Then, $s_1$ might be a better strategy than $s_2$ if $var[T_{s_1}] > var[T_{s_2}]$. However, this is simply an intuition and has to be proven and/or tested. Lets simply focus on finding $s$ which minimises $E[T_s]$ for now. 



Therefore, we can represent the state of the entire board as a row vector $B = (X^0,...,X^{99})$, where $X^i = 1$ if the $i\textsuperscript{th}$ tile has a ship and $X^i = 0$ if it doesn't.

Note we make no i.i.d assumptions for $X^i$. Since, for example, if we know tile i has a ship, then we can conclude that adjacent tiles are more likely to be ships as well.

TODO: TALK ABOUT SAMPLE SPACE, FINITE, ENUMERATE OVER ALL

\begin{thebibliography}{11}

\bibitem{rochford} https://austinrochford.com/posts/2021-09-02-battleship-bayes.html
\bibitem{alemi} http://thevirtuosi.blogspot.com/2011/10/linear-theory-of-battleship.html
\end{thebibliography}
\end{document}

Now, $X^i \sim Ber(\theta_i)$, and the maximum likelihood estimator for Bernoulli is:
$$ \theta_{MLE} = \frac{1}{n} \sum_{j=1}^{n} x^i_j $$ where $$
* TODO: TALK ABOUT SAMPLE SPACE, FINITE, ENUMERATE OVER ALL
* Noticed that al lot of algorithms can be done have an increased variance meaning better play