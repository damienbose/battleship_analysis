\documentclass[11pt]{article}
\usepackage[left=3cm,right=3cm,top=3cm,bottom=3cm]{geometry}
\usepackage{mathtools}
\usepackage{multicol}

%% (other macros, packages and definitions go here)

\newtheorem{definition}{Definition}

\begin{document}

\title{Analysis of the Battleship Game}
\author{Damien R. Bose}
\date{21/02/2023}

\maketitle

\section{Introduction}

This article will try and analyse the game of battleship. Our goal is to come up with a strategy for optimal (or close to optimal) play, inspired by an article by Anthony Rochford's analysis\cite{rochford}. However, I try to formalise these ideas in my own way. Furthermore, I try pay more attention to about the behaviour of the opponent. 

\section{Rules of the Game\cite{rules}}

\section{Mathematical Setup}
\subsection{Modelling the Game}

We first come to the question of modelling the opponents board. First, however, notice that the game lies on a $10 \times 10$ grid. Hence, there are 100 tiles. Therefore, we can model them as a series of random variables.

\begin{definition}
$X^i$ denotes wether the $i\textsuperscript{th}$ tile is a ship or not. $X^i = 1$ if the $i\textsuperscript{th}$ tile is a ship and $X^i = 0$ if it doesn't.
\end{definition}

$X^i \sim Ber(\theta^i)$. Hence $P(X^i = 1) = \theta^i$ is the probability that tile $i$ has a boat.

Notice that we make no i.i.d assumptions for $X^i$. Since, for example, if we know tile $i$ has a ship, then we can conclude that adjacent tiles are more likely to be ships as well. In fact, conditioning on previous knowledge will guide our strategy.

\subsection{Modelling Performance}
Now lets not forget out goal: be the first to hit all 17 boat tiles of the opponent. One way to this is by minimising the time it takes to hit all ships.


\begin{definition}
 $T_s$ is the number of moves/time it takes for us to do so with strategy s.
\end{definition}

We can therefore quantify the performance of our strategy using $E[T_s]$.

Note, however, $E[T_s]$ does not capture the entire picture regarding the success of a strategy: winning is dependent on the performance of the other player as well. For example, let's say player 1 has strategy $s_1$ and player 2 has strategy $s_2$, and $E[T_{s_1}] = E[T_{s_2}]$. Then, $s_1$ might be a better strategy than $s_2$ if $var[T_{s_1}] > var[T_{s_2}]$. However, this is simply an intuition and has to be proven and/or tested. Lets simply focus on finding $s$ which minimises $E[T_s]$ for now. 

\section{Strategy}
Now that we have the mathematical setup, we can start exploring our strategy. 

During turn 1, we have no information about the opponents board. So we simply pick the tile i which maximises $P(X_i = 1)$. Lets say tile 55 gets picked. We find out that $x_i = 1$. Hit! Now, we pick the tile $ j \neq i$ which maximises $P(X_j = 1 | X_i = 1)$, but now $x_j = 0$. We missed. We now pick tile k with max $P(X_k = 1 | X_j = 0, X_i = 1)$, and so on iteratively. 

A possible improvement to this strategy is instead of picking i which maximises the probability of that tile having a ship, we should pick the our next target $j$ randomly with corresponding probability $P(X_j = 1 | ...)$. This likely has an equal $E[T_s]$ to the initial strategy but will likely have a higher $var[T_s]$.


\begin{thebibliography}{11}
\bibitem{rules} https://www.hasbro.com/common/instruct/battleship.pdf
\bibitem{rochford} https://austinrochford.com/posts/2021-09-02-battleship-bayes.html
\bibitem{alemi} http://thevirtuosi.blogspot.com/2011/10/linear-theory-of-battleship.html
\end{thebibliography}
\end{document}

\subsection{Limitations of $T_s$}

%TODO: ASK PROF ABOUT THE FIRST
%TODO: ALL POSSIBLE ENUMERATIONS
%TODO: RANDOM SAMPLING
%TODO: GUIDED SEARCH USING BASIAN TO HAVE RANDOM BUT ALSO USE SOMETHING ELSE TO LEARN OPPONENT PLACMENT PATTERN

%First, the opponent is responsible for placing the initial ships; hence they determine our initial $\theta^i$ parameters. (TALK about when trying to determin theta)

%-----------
%Now, importantly we must remember that the opponent is responsible for initially placing their ships, so the opponents placement strategy will determine the value of these $\theta^i$ parameters. Other papers\cite{rochford}\cite{alemi} assume that the ships are randomly distributed. However, a more intelligent opponent (who has information about our strategy) might not do that. 
%
%----------
%Now, $X^i \sim Ber(\theta_i)$, and the maximum likelihood estimator for Bernoulli is:
%$$ \theta_{MLE} = \frac{1}{n} \sum_{j=1}^{n} x^i_j $$ where $$
%* TODO: TALK ABOUT SAMPLE SPACE, FINITE, ENUMERATE OVER ALL
%* Noticed that al lot of algorithms can be done have an increased variance meaning better play
